\setdefaultlanguage{turkish}  % Ana dili Türkçe olarak ayarlanır (polyglossia ile birlikte kullanılmalı)

% ---------- ÖZ ----------
\chapter*{ÖZ}  % Numarasız bölüm başlığı
\addcontentsline{toc}{chapter}{ÖZ}  % İçindekilere "ÖZ" başlığını ekle
Bu çalışma...

% ---------- ABSTRACT ----------
\chapter*{ABSTRACT}
\addcontentsline{toc}{chapter}{ABSTRACT}
This study...

% ---------- ÖNSÖZ ----------
\chapter*{ÖNSÖZ}
\addcontentsline{toc}{chapter}{ÖNSÖZ}
Bu tez çalışması süresince...

\newpage  % Sonraki sayfaya geç

% ---------- İÇİNDEKİLER ----------
\setcounter{tocdepth}{2}  % İçindekilerde sadece chapter, section ve subsection göster (4 yerine 2 önerilir)
\tableofcontents  % İçindekiler

% Başlık ve sayfa numarası yazı tiplerini özelleştir
\renewcommand{\cftchapfont}{\bfseries}       % Chapter başlıkları kalın
\renewcommand{\cftsecfont}{\bfseries}        % Section başlıkları kalın
\renewcommand{\cftsubsecfont}{\normalfont}   % Subsection başlıkları normal
\renewcommand{\cftchappagefont}{\bfseries}   % Sayfa numaralarını da kalın yap
\renewcommand{\cftsecpagefont}{\bfseries}

\newpage

% ---------- ŞEKİLLER LİSTESİ ----------
\listoffigures  % Eğer belgede \begin{figure} varsa bu liste oluşur

% ---------- TABLOLAR LİSTESİ ----------
\listoftables   % Eğer belgede \begin{table} varsa bu liste oluşur

\newpage

% ---------- KISALTMALAR ----------
\chapter*{KISALTMALAR}
\addcontentsline{toc}{chapter}{KISALTMALAR}
\begin{description}[leftmargin=4cm]  % Genişlik ayarıyla hizalama düzgün olur
  \item[T.C.] Türkiye Cumhuriyeti
  \item[vd.] ve diğerleri
\end{description}
