\documentclass[12pt,a4paper,oneside,shorthands=:!]{iusosbil}
% Özel sınıf dosyasını (iusosbil.cls) temel alır.
% 12pt        : Yazı boyutu
% a4paper     : A4 kağıt boyutu
% oneside     : Tek yüzlü yazım
% turkish     : Ana dil Türkçe
% shorthands  : Bazı kısayolları devre dışı bırakır (özellikle ":" karakteri için)

% ---------- Paketler ----------
%\geometry{a4paper, margin=2.5cm} 
% Sayfa kenar boşluklarını 2.5 cm olarak ayarlar (bu satır sınıf dosyasındaki ayarları geçersiz kılar).

\usepackage{fontspec}         % Sistem yazı tiplerini kullanmak için (XeLaTeX ile zorunlu).
\usepackage{polyglossia}      % Çok dilli destek sağlar.
\setdefaultlanguage{turkish}  % Ana dili Türkçe olarak ayarlar (tireleme, tarih biçimi, çeviriler vs.).

% Biçimlendirme ve başlıklar
\usepackage{titlesec}         % Başlıkları biçimlendirmek için kullanılır (chapter, section, vs.).
\titleformat{\chapter}{\centering\bfseries}{\thechapter.}{1em}{}% Chapter başlıklarını ortalar, kalın yapar ve başına numara koyar.
\titleformat{\section}{\bfseries}{\thesection}{1em}{}% Section başlıklarını kalın ve numaralı yapar.
% İçindekiler özelleştirmesi
\usepackage{tocloft}  % İçindekiler, şekil ve tablo listelerini özelleştirmek için.
% İçindekiler'de nokta çizgisi aktif
\renewcommand{\cftdot}{.}         % Noktaları aktif et
\setlength{\cftbeforesecskip}{0.5em}% İçindekilerde bölüm başlıkları arasında dikey boşluk.
\renewcommand{\cftchapfont}{\bfseries}% İçindekilerde bölüm başlıklarını kalın yapar.
\renewcommand{\cftsecfont}{\rmfamily}
\renewcommand{\cftsubsecfont}{\rmfamily}% İçindekilerde alt başlıkları normal yazı tipiyle gösterir.
\renewcommand{\cftchappagefont}{\bfseries}
\renewcommand{\cftsecpagefont}{\rmfamily}
\renewcommand{\cftsubsecpagefont}{\rmfamily}% Sayfa numaralarının yazı tipini belirler.
\usepackage{ragged2e}
% Paragrafları iki yana yaslamak için kullanılır (özellikle \justifying komutuyla birlikte).
% Sayfa düzeni ve üstbilgi/altbilgi → sınıf dosyasından gelmiş olabilir, burada tanımlı değil.

% Kaynakça yönetimi
\usepackage[backend=biber,style=authoryear,sorting=nyt]{biblatex}
% Kaynakça paketi: 
% - biber motoru ile çalışır, 
% - yazar-tarih stilini kullanır, 
% - kaynakları yazar > yıl > başlığa göre sıralar.

\addbibresource{kaynakca.bib} % Kaynakların bulunduğu .bib dosyasını ekler.
% Diğer yardımcı paketler
\usepackage{hyperref} % PDF içi bağlantılar (içindekilerden tıklama gibi), URL'ler vb.
\usepackage{microtype}% Tipografik düzeltmeler (karakter aralıkları, taşmalar vs.) için.
\usepackage{bookmark} % PDF yer imlerini düzgün oluşturur (hyperref ile birlikte çalışır).
\usepackage{caption} % Şekil ve tablo başlıklarını özelleştirmek için.
\usepackage{enumitem} % Liste yapılarının (itemize, enumerate) biçimini özelleştirmek için.
\usepackage{lipsum} % Sahte metin üretmek için (tasarım/test amaçlıdır; yayında çıkarılmalı).
\usepackage{setspace}
\setstretch{1.5} % Satır aralığını 1.5 yapar (tez formatı için yaygın).

% ---------- Belge Başlangıcı ----------
\begin{document}
\justifying     % Paragrafları iki yana yasla (ragged2e paketi ile çalışır)
%----------- İÇ KAPAK -----------
\begin{titlepage}
\begin{center}

    \bfseries
    \fontsize{14pt}{16pt}\selectfont

    T.C. \\
    İSTANBUL ÜNİVERSİTESİ \\
    SOSYAL BİLİMLER ENSTİTÜSÜ \\
    \MakeUppercase{ANABİLİM DALI ADI} \\
    \MakeUppercase{BİLİM DALI ADI} \\
    \vspace{2.5cm}
    \MakeUppercase{DOKTORA TEZİ} \\
    

    \vspace{2.5cm}
    \MakeUppercase{TEZ BAŞLIĞI} \\

    \vspace{2.5cm}

    Adı \textsc{SOYADI} \\ % Öğrenci adı ve SOYADI
    12345678 \\

    \vfill
    \fontsize{12pt}{14pt}\selectfont

    TEZ DANIŞMANI: \\
    Unvan Adı \textsc{SOYADI} \\ % Danışman adı ve SOYADI

    \vspace{1cm}

    \centering
    \fontsize{14pt}{16pt}\selectfont
    İSTANBUL – 20XX

\end{center}
\end{titlepage}
% kapak.tex dosyasını dahil eder (başlık, öğrenci adı, enstitü vs.)
\pagenumbering{roman}% Sayfa numaralarını Roma rakamı olarak başlatır (I, II, III, …)
\setcounter{page}{3} % İlk iki sayfa (kapak, onay) numarasız varsayılıyorsa, gerçek numarayı 3'ten başlatır
\setdefaultlanguage{turkish}  % Ana dili Türkçe olarak ayarlanır (polyglossia ile birlikte kullanılmalı)

% ---------- ÖZ ----------
\chapter*{ÖZ}  % Numarasız bölüm başlığı
\addcontentsline{toc}{chapter}{ÖZ}  % İçindekilere "ÖZ" başlığını ekle
Bu çalışma...

% ---------- ABSTRACT ----------
\chapter*{ABSTRACT}
\addcontentsline{toc}{chapter}{ABSTRACT}
This study...

% ---------- ÖNSÖZ ----------
\chapter*{ÖNSÖZ}
\addcontentsline{toc}{chapter}{ÖNSÖZ}
Bu tez çalışması süresince...

\newpage  % Sonraki sayfaya geç

% ---------- İÇİNDEKİLER ----------
\setcounter{tocdepth}{2}  % İçindekilerde sadece chapter, section ve subsection göster (4 yerine 2 önerilir)
\tableofcontents  % İçindekiler

% Başlık ve sayfa numarası yazı tiplerini özelleştir
\renewcommand{\cftchapfont}{\bfseries}       % Chapter başlıkları kalın
\renewcommand{\cftsecfont}{\bfseries}        % Section başlıkları kalın
\renewcommand{\cftsubsecfont}{\normalfont}   % Subsection başlıkları normal
\renewcommand{\cftchappagefont}{\bfseries}   % Sayfa numaralarını da kalın yap
\renewcommand{\cftsecpagefont}{\bfseries}

\newpage

% ---------- ŞEKİLLER LİSTESİ ----------
\listoffigures  % Eğer belgede \begin{figure} varsa bu liste oluşur

% ---------- TABLOLAR LİSTESİ ----------
\listoftables   % Eğer belgede \begin{table} varsa bu liste oluşur

\newpage

% ---------- KISALTMALAR ----------
\chapter*{KISALTMALAR}
\addcontentsline{toc}{chapter}{KISALTMALAR}
\begin{description}[leftmargin=4cm]  % Genişlik ayarıyla hizalama düzgün olur
  \item[T.C.] Türkiye Cumhuriyeti
  \item[vd.] ve diğerleri
\end{description}
% önkısım.tex: Teşekkür, özet, abstract, içindekiler vb. yer alır
\pagenumbering{arabic}% Sayfa numaralarını normal (1, 2, 3, …) olarak sıfırlar
\setcounter{page}{1}% Ana metin (giriş bölümü) 1. sayfadan başlar
\chapter*{GİRİŞ}
	\lipsum[1-30]
 % bölüm_giriş.tex dosyasını dahil eder
\chapter{}
\section{vdvfdbdbdf}
\lipsum[1-20]
\subsection{} % bölüm_1.tex dosyasını dahil eder (1. bölüm)
\chapter{ }
\lipsum % bölüm_2.tex dosyasını dahil eder (2. bölüm)
\chapter{ } % bölüm_3.tex dosyasını dahil eder (3. bölüm)
\chapter*{SONUÇ}
\lipsum % bölüm_sonuç.tex dosyasını dahil eder
\include{kaynakça} % kaynakça.tex dosyasını dahil eder (çoğunlukla \printbibliography içerir)
\chapter*{EKLER}
Ekler burada yer alır. % bölüm_ekler.tex dosyasını dahil eder (ekler, tablolar, görseller vs.)

\end{document}
